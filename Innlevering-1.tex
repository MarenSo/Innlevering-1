% Options for packages loaded elsewhere
\PassOptionsToPackage{unicode}{hyperref}
\PassOptionsToPackage{hyphens}{url}
%
\documentclass[
]{article}
\usepackage{amsmath,amssymb}
\usepackage{lmodern}
\usepackage{ifxetex,ifluatex}
\ifnum 0\ifxetex 1\fi\ifluatex 1\fi=0 % if pdftex
  \usepackage[T1]{fontenc}
  \usepackage[utf8]{inputenc}
  \usepackage{textcomp} % provide euro and other symbols
\else % if luatex or xetex
  \usepackage{unicode-math}
  \defaultfontfeatures{Scale=MatchLowercase}
  \defaultfontfeatures[\rmfamily]{Ligatures=TeX,Scale=1}
\fi
% Use upquote if available, for straight quotes in verbatim environments
\IfFileExists{upquote.sty}{\usepackage{upquote}}{}
\IfFileExists{microtype.sty}{% use microtype if available
  \usepackage[]{microtype}
  \UseMicrotypeSet[protrusion]{basicmath} % disable protrusion for tt fonts
}{}
\makeatletter
\@ifundefined{KOMAClassName}{% if non-KOMA class
  \IfFileExists{parskip.sty}{%
    \usepackage{parskip}
  }{% else
    \setlength{\parindent}{0pt}
    \setlength{\parskip}{6pt plus 2pt minus 1pt}}
}{% if KOMA class
  \KOMAoptions{parskip=half}}
\makeatother
\usepackage{xcolor}
\IfFileExists{xurl.sty}{\usepackage{xurl}}{} % add URL line breaks if available
\IfFileExists{bookmark.sty}{\usepackage{bookmark}}{\usepackage{hyperref}}
\hypersetup{
  pdftitle={R Notebook},
  hidelinks,
  pdfcreator={LaTeX via pandoc}}
\urlstyle{same} % disable monospaced font for URLs
\usepackage[margin=1in]{geometry}
\usepackage{graphicx}
\makeatletter
\def\maxwidth{\ifdim\Gin@nat@width>\linewidth\linewidth\else\Gin@nat@width\fi}
\def\maxheight{\ifdim\Gin@nat@height>\textheight\textheight\else\Gin@nat@height\fi}
\makeatother
% Scale images if necessary, so that they will not overflow the page
% margins by default, and it is still possible to overwrite the defaults
% using explicit options in \includegraphics[width, height, ...]{}
\setkeys{Gin}{width=\maxwidth,height=\maxheight,keepaspectratio}
% Set default figure placement to htbp
\makeatletter
\def\fps@figure{htbp}
\makeatother
\setlength{\emergencystretch}{3em} % prevent overfull lines
\providecommand{\tightlist}{%
  \setlength{\itemsep}{0pt}\setlength{\parskip}{0pt}}
\setcounter{secnumdepth}{-\maxdimen} % remove section numbering
\ifluatex
  \usepackage{selnolig}  % disable illegal ligatures
\fi

\title{R Notebook}
\author{}
\date{\vspace{-2.5em}}

\begin{document}
\maketitle

\hypertarget{header-1-minst-4-overskrifter}{%
\section{Header 1 (minst 4
overskrifter)}\label{header-1-minst-4-overskrifter}}

\hypertarget{header-2}{%
\subsection{Header 2}\label{header-2}}

\hypertarget{header-3}{%
\subsubsection{Header 3}\label{header-3}}

\hypertarget{ordnet-liste-minst-1}{%
\subsubsection{Ordnet liste (minst 1):}\label{ordnet-liste-minst-1}}

\begin{enumerate}
\def\labelenumi{\arabic{enumi}.}
\item
  Første

  \begin{itemize}
  \tightlist
  \item
    hei
  \end{itemize}
\item
  Andre
\end{enumerate}

\begin{itemize}
\tightlist
\item
  hei
\end{itemize}

Minst et eksempel på bruk av følgende: \emph{italic} \textbf{bold}
\textbf{\emph{italicbold}}

\hypertarget{reproduserbarhet}{%
\section{Reproduserbarhet}\label{reproduserbarhet}}

I 2015 oppstod det en replikasjonskrise. Dette startet innenfor
psykologien, hvor 270 forskere samarbeidet om å forsøke å replikere 100
studier som alle var publisert i ledende tidsskrifter innenfor
fagfeltet. De klarte kun å få samme resultat i under halvparten av
studiene, og dette var selv med hjelp fra forskerne som stod bak disse
studiene
(\url{https://psykologtidsskriftet.no/fagessay/2017/07/replikasjonskrisen}).
I ettertid har det blitt større fokus på replikerbarhet og
reproduserbarhet innenfor forskning. Replikerbarhet er et krav for
reproduserbarhet, og i denne oppgaven skal jeg ta for meg
reproduserbarhet og om bruk av R og R notebooks kan være en mulig
løsning for å gjøre forskning reproduserbar.

Det er enda ingen allmenn akseptert definisjon av ``reproduserbarhet''
og ``replikerbarhet''. Noen bruker disse begrepene om hverandre (Bollen
et al., 2015), og andre er nøye med å skille dem fra hverandre (Leek and
Peng, 2015: Goodman et al., 2016). I denne oppgaven vil vi skille dem
fra hveraandre og bruke Bollens (et al., 2015) definisjoner av
begrepene: ``reproduserbarhet'' oppstår dersom forskere klarer å komme
frem til samme resultat ved å bruke samme prosedyre og samme datasett
som gjort ved det opprinnelige studiet. ``Replikerbarhet'' oppstår
dersom forskere klarer å komme frem til samme resultatet ved å bruke
samme prosedyre og et nytt datasett. Hovedforskjellen er altså at ved
``replikerbarhet'' så skal det hentes inn nye data, men resultatet skal
likevel bli det samme. De som prøver å replikere eller reprodusere
studiet må altså ha tilgang til alt av data, kildekode og
prosedyredetaljer. Man kan dermed si at reproduserbarhet er et krav for
å kunne oppnå replikerbarhet.

I forskning er det et problem at studier ikke kan reproduseres. Selv med
samme data som er brukt i studiet, tilgjengelig vil resultatet sjeldent
bli det samme. I denne oppgaven skal vi ta for oss om R notebook kan
være med på å blabla

\end{document}
