% Options for packages loaded elsewhere
\PassOptionsToPackage{unicode}{hyperref}
\PassOptionsToPackage{hyphens}{url}
%
\documentclass[
  12pt,
]{article}
\usepackage{amsmath,amssymb}
\usepackage{lmodern}
\usepackage{ifxetex,ifluatex}
\ifnum 0\ifxetex 1\fi\ifluatex 1\fi=0 % if pdftex
  \usepackage[T1]{fontenc}
  \usepackage[utf8]{inputenc}
  \usepackage{textcomp} % provide euro and other symbols
\else % if luatex or xetex
  \usepackage{unicode-math}
  \defaultfontfeatures{Scale=MatchLowercase}
  \defaultfontfeatures[\rmfamily]{Ligatures=TeX,Scale=1}
\fi
% Use upquote if available, for straight quotes in verbatim environments
\IfFileExists{upquote.sty}{\usepackage{upquote}}{}
\IfFileExists{microtype.sty}{% use microtype if available
  \usepackage[]{microtype}
  \UseMicrotypeSet[protrusion]{basicmath} % disable protrusion for tt fonts
}{}
\makeatletter
\@ifundefined{KOMAClassName}{% if non-KOMA class
  \IfFileExists{parskip.sty}{%
    \usepackage{parskip}
  }{% else
    \setlength{\parindent}{0pt}
    \setlength{\parskip}{6pt plus 2pt minus 1pt}}
}{% if KOMA class
  \KOMAoptions{parskip=half}}
\makeatother
\usepackage{xcolor}
\IfFileExists{xurl.sty}{\usepackage{xurl}}{} % add URL line breaks if available
\IfFileExists{bookmark.sty}{\usepackage{bookmark}}{\usepackage{hyperref}}
\hypersetup{
  pdftitle={R Notebook og reproduserbarhet},
  pdfauthor={Innlevering 1 i Data Science 2021 - Maren Sognefest og Daniel Karstad},
  hidelinks,
  pdfcreator={LaTeX via pandoc}}
\urlstyle{same} % disable monospaced font for URLs
\usepackage[margin=1in]{geometry}
\usepackage{color}
\usepackage{fancyvrb}
\newcommand{\VerbBar}{|}
\newcommand{\VERB}{\Verb[commandchars=\\\{\}]}
\DefineVerbatimEnvironment{Highlighting}{Verbatim}{commandchars=\\\{\}}
% Add ',fontsize=\small' for more characters per line
\usepackage{framed}
\definecolor{shadecolor}{RGB}{248,248,248}
\newenvironment{Shaded}{\begin{snugshade}}{\end{snugshade}}
\newcommand{\AlertTok}[1]{\textcolor[rgb]{0.94,0.16,0.16}{#1}}
\newcommand{\AnnotationTok}[1]{\textcolor[rgb]{0.56,0.35,0.01}{\textbf{\textit{#1}}}}
\newcommand{\AttributeTok}[1]{\textcolor[rgb]{0.77,0.63,0.00}{#1}}
\newcommand{\BaseNTok}[1]{\textcolor[rgb]{0.00,0.00,0.81}{#1}}
\newcommand{\BuiltInTok}[1]{#1}
\newcommand{\CharTok}[1]{\textcolor[rgb]{0.31,0.60,0.02}{#1}}
\newcommand{\CommentTok}[1]{\textcolor[rgb]{0.56,0.35,0.01}{\textit{#1}}}
\newcommand{\CommentVarTok}[1]{\textcolor[rgb]{0.56,0.35,0.01}{\textbf{\textit{#1}}}}
\newcommand{\ConstantTok}[1]{\textcolor[rgb]{0.00,0.00,0.00}{#1}}
\newcommand{\ControlFlowTok}[1]{\textcolor[rgb]{0.13,0.29,0.53}{\textbf{#1}}}
\newcommand{\DataTypeTok}[1]{\textcolor[rgb]{0.13,0.29,0.53}{#1}}
\newcommand{\DecValTok}[1]{\textcolor[rgb]{0.00,0.00,0.81}{#1}}
\newcommand{\DocumentationTok}[1]{\textcolor[rgb]{0.56,0.35,0.01}{\textbf{\textit{#1}}}}
\newcommand{\ErrorTok}[1]{\textcolor[rgb]{0.64,0.00,0.00}{\textbf{#1}}}
\newcommand{\ExtensionTok}[1]{#1}
\newcommand{\FloatTok}[1]{\textcolor[rgb]{0.00,0.00,0.81}{#1}}
\newcommand{\FunctionTok}[1]{\textcolor[rgb]{0.00,0.00,0.00}{#1}}
\newcommand{\ImportTok}[1]{#1}
\newcommand{\InformationTok}[1]{\textcolor[rgb]{0.56,0.35,0.01}{\textbf{\textit{#1}}}}
\newcommand{\KeywordTok}[1]{\textcolor[rgb]{0.13,0.29,0.53}{\textbf{#1}}}
\newcommand{\NormalTok}[1]{#1}
\newcommand{\OperatorTok}[1]{\textcolor[rgb]{0.81,0.36,0.00}{\textbf{#1}}}
\newcommand{\OtherTok}[1]{\textcolor[rgb]{0.56,0.35,0.01}{#1}}
\newcommand{\PreprocessorTok}[1]{\textcolor[rgb]{0.56,0.35,0.01}{\textit{#1}}}
\newcommand{\RegionMarkerTok}[1]{#1}
\newcommand{\SpecialCharTok}[1]{\textcolor[rgb]{0.00,0.00,0.00}{#1}}
\newcommand{\SpecialStringTok}[1]{\textcolor[rgb]{0.31,0.60,0.02}{#1}}
\newcommand{\StringTok}[1]{\textcolor[rgb]{0.31,0.60,0.02}{#1}}
\newcommand{\VariableTok}[1]{\textcolor[rgb]{0.00,0.00,0.00}{#1}}
\newcommand{\VerbatimStringTok}[1]{\textcolor[rgb]{0.31,0.60,0.02}{#1}}
\newcommand{\WarningTok}[1]{\textcolor[rgb]{0.56,0.35,0.01}{\textbf{\textit{#1}}}}
\usepackage{graphicx}
\makeatletter
\def\maxwidth{\ifdim\Gin@nat@width>\linewidth\linewidth\else\Gin@nat@width\fi}
\def\maxheight{\ifdim\Gin@nat@height>\textheight\textheight\else\Gin@nat@height\fi}
\makeatother
% Scale images if necessary, so that they will not overflow the page
% margins by default, and it is still possible to overwrite the defaults
% using explicit options in \includegraphics[width, height, ...]{}
\setkeys{Gin}{width=\maxwidth,height=\maxheight,keepaspectratio}
% Set default figure placement to htbp
\makeatletter
\def\fps@figure{htbp}
\makeatother
\setlength{\emergencystretch}{3em} % prevent overfull lines
\providecommand{\tightlist}{%
  \setlength{\itemsep}{0pt}\setlength{\parskip}{0pt}}
\setcounter{secnumdepth}{-\maxdimen} % remove section numbering
\ifluatex
  \usepackage{selnolig}  % disable illegal ligatures
\fi

\title{R Notebook og reproduserbarhet}
\author{Innlevering 1 i Data Science 2021 - Maren Sognefest og Daniel
Karstad}
\date{}

\begin{document}
\maketitle

\begin{Shaded}
\begin{Highlighting}[]
\FunctionTok{plot}\NormalTok{(cars)}
\end{Highlighting}
\end{Shaded}

\includegraphics{Innlevering-1_files/figure-latex/unnamed-chunk-1-1.pdf}

Vi mangler:

Kjør sessioninfo

\begin{enumerate}
\def\labelenumi{\arabic{enumi})}
\setcounter{enumi}{3}
\item
  Minst 1 internt bilde skal være screenshot av git history som:
\item
  Dokumenterer minst 10 «commits»2) Dokumenterer bruk av minst 3
  «branches»3) Ekstra stjerne til dem som klarer å få til en «merge
  conflict» ;-)4) Bildet som dokumenterer git history skal være i et
  appendiks som kommer helttil slutt i dokumentet (etter referansene)5)
  KjørsessionInfo()i en code-chunk (husk å gi chunk-en navn). Hvordan
  kan dennefunksjonen hjelpe oss med å gjøre et dokument
  reproduserbart?6) Vi benytter apa for sitering og referanseliste
  (apa-no-ampersand.csler tilgjengeligunderFileri Canvas.)
\end{enumerate}

Bruk begge siteringsformene, dvs med og uten{[}{]}1) Husk at for å få
siteringsinfo for R pakker kan dere bruke kommandoento
Bibtex(citation(\textless navn-R-pakke\textgreater)), f.eks

Vi mangler flere kilder, også i teksten

Minst et eksempel på bruk av følgende: \textbf{bold}
\textbf{\emph{italicbold}}

\hypertarget{reproduserbarhet}{%
\section{Reproduserbarhet}\label{reproduserbarhet}}

I senere tid har det oppstått en replikasjonskrise innenfor forskning.
Denne startet innenfor psykologien, og ble for alvor offentlig kjent i
2015 da 270 forskere samarbeidet om å forsøke å replikere 100 studier
som alle var publisert i ledende tidsskrifter innenfor fagfeltet. Da de
forsøkte å replikere studiene, klarte de kun å få samme resultat i under
halvparten av studiene, og dette var med hjelp fra forskerne som stod
bak disse studiene (Sætrevik B., 2017). Det har senere vist seg at disse
replikasjonsproblemene finnes innenfor flere fagfelt, og i ettertid har
det blitt større fokus på reproduserbarhet innenfor forskning.
Reproduserbarhet er en forutsetning for replikerbarhet, så denne
oppgaven skal vi ta for oss reproduserbarhet og hvorvidt bruk av R og R
notebooks kan være en mulig løsning for å gjøre forskning reproduserbar,
og dermed mer pålitelig.

\hypertarget{litteraturgjennomgang}{%
\subsection{Litteraturgjennomgang}\label{litteraturgjennomgang}}

Det er enda ingen allmenn definisjon av ``reproduserbarhet'' og
``replikerbarhet.'' Noen bruker disse begrepene om hverandre (Bollen et
al., 2015), og andre er nøye med å skille dem fra hverandre (Leek and
Peng, 2015: Goodman et al., 2016). I denne oppgaven vil vi skille
begrepene tydelig fra hverandre og bruke Bollens (et al., 2015)
definisjoner av begrepene: ``reproduserbarhet'' oppstår dersom forskere
klarer å komme frem til samme resultat ved å bruke samme prosedyre og
samme datasett som gjort ved det opprinnelige studiet.
``Replikerbarhet'' oppstår dersom forskere klarer å komme frem til samme
resultatet ved å bruke samme prosedyre og et nytt datasett.
Hovedforskjellen er altså at ved ``replikerbarhet'' så skal det hentes
inn nye data, men resultatet skal likevel bli det samme. De som prøver å
replikere eller reprodusere studiet må altså ha tilgang til alt av data,
kildekode og prosedyredetaljer. Man kan dermed si at reproduserbarhet er
en betingelse for å kunne oppnå replikerbarhet.

\hypertarget{problemets-omfang}{%
\subsubsection{Problemets omfang}\label{problemets-omfang}}

I lys av replikasjonskrisen og det økte fokuset på replikasjon, har
flere tidsskrifter begynt å publisere tilhørende datasett sammen med
artiklene, men det er flere forskere o.l. som ikke ønsker å gi fra seg
denne informasjonen. Dette gjelder data de besitter, koder,
fremgangsmåte, dokumentasjon, resultat, feil, problemer de har møtt på,
hypoteser osv. Dette gjør det svært vanskelig å reprodusere en tidligere
studie, på et senere tidspunkt. Siden det samlet er flere problemer, vil
det naturligvis også være flere løsninger som må tas i bruk for at full
reproduksjon skal være oppnåelig og man skal komme frem til lignende
konklusjoner med ny data og ny sammensetning.

Vi kan definere og skille mellom tekniske og menneskelige løsninger. Det
menneskelige aspektet i problemstillingen er ofte knyttet til det
forskerne selv velger å dele av data, informasjon, koder, hypoteser,
fremgangsmåte, programvare og så videre. Det har ikke vært praktisert,
og standard retningslinjer for hva som bør ansees som god
forskningsskikk og praksis er nødt å komme på plass for å møte kravene
om tilfredsstillende reproduksjon og replikasjon. Det tekniske aspektet
byr på mangel av data, koder, feil i programvare og feil som har
oppstått underveis. Ved at man kan integrere og implementere programkode
hos tidsskriftene, synlig eller usynlig, så skal det være mulig for
andre forskere å reprodusere studien og gjøre den replikerbar.

\hypertarget{oversikt-over-hva-som-buxf8r-sendes-til-tidsskriftene-for-uxe5-gjuxf8re-studien-replikerbar}{%
\paragraph{Oversikt over hva som bør sendes til tidsskriftene for å
gjøre studien
replikerbar:}\label{oversikt-over-hva-som-buxf8r-sendes-til-tidsskriftene-for-uxe5-gjuxf8re-studien-replikerbar}}

\begin{itemize}
\tightlist
\item
  En kode til å kunne lese inn dataen med
\item
  En kode til å kalkulere og analysere dataen
\item
  En kode for å teste i henhold til hypoetese
\item
  En kode for å generere en rapport av resultatet
\end{itemize}

\hypertarget{mulig-luxf8sning-teoretisk-plan-compendium-dynamic-document-code-chuncks-og-text-chunck}{%
\subsubsection{Mulig løsning (teoretisk plan) ``Compendium,'' ``Dynamic
document,'' ``code chuncks'' og ``text
chunck''}\label{mulig-luxf8sning-teoretisk-plan-compendium-dynamic-document-code-chuncks-og-text-chunck}}

I henhold til Gentleman og Lang (Kilde) er det nøye å integrere koder og
beregninger som blir brukt i dataanalyser, metodebeskrivelser og
simuleringer. Dette kan enkelt gjøres via et kompendium, hevdet av
Gentleman og Lang (2007). Kompendium er en kortfattet oversikt over
hovedinnholdet i f.eks. en studie gitt i dette tilfellet. Kompendiumet
til da gi en oversikt over innholdet, slik som tekst, kode, data,
metodikk, hypotese, problemstilling og så videre. Dette gjør at
kompendiumet enkelt kan distribueres i ulike kanaler, enkelt kan
håndteres og oppdateres i henhold.

Før var det RMarkdown som oftest ble brukt. Problemet der var at man
ofte ikke fikk all tekst, data og koding i samme dokument, man måtte
dele det opp i ulike tabs. RNotebook er den nyeste utgivelsen fra
Rstudio.\\
Rstudio er en Integrated Developer Environment (IDE) for alt som er R
relatert. Rstudio er gratis, både å laste ned og gratis å bruke. Man kan
laste det ned lokalt på PC/Desktop eller jobbe online/remote. Alle
vanlige operativ system (OS) skal være kompatible til å bruke Rstudio,
bla. Mac, Windows, Linux. Rstudio må benyttes sammen med andre program
og/eller extensions, f.eks GitAhead, GitHub Desktop,
kommandolinje/terminal, for å kunne oppnå reproduserbarhet og
replikerbarhet. Dette oppnås f.eks med kodeversjonskontroll, f.eks Git
koblet med Github. En RNotebook er et RMarkdown dokument som inneholder
kode+tekst-blokker, som henter inn koder og data, utfører beregninger og
analyser i henhold til formler/kode som legges inn i Rmd filen.
RNotebook vil da kunne vise oss et ferdig, vanlig tekstdokument, som
inneholder tekst og koder for relevant innhold, istedenfor å ha dette i
flere forskjellige filer. Dette gjør at du visuelt kan vurdere dataene
mens du utvikler RMarkdown dokumentet uten å måtte «knyte» sammen hele
dokumentet for å se resultatet.

Dette gjør at RNotebook kan brukes til å løse problemer knyttet til
reproduserbarhet og replikerbarhet.

\hypertarget{vil-dagens-luxf8sning-med-arkiv-av-data-og-eventuell-programkode-hos-tidsskriftene-kunne-luxf8se-problemet}{%
\subsubsection{Vil dagens løsning med arkiv av data og eventuell
programkode hos tidsskriftene kunne løse
problemet?}\label{vil-dagens-luxf8sning-med-arkiv-av-data-og-eventuell-programkode-hos-tidsskriftene-kunne-luxf8se-problemet}}

En mulig løsning på problemet kan være å publisere forskningsartikler i
kompendier, som også inneholder datasett og koder som er brukt i
forskningen. I et slikt kompendium kan det være dokumenter som kan
oppdateres, også kjent som dynamiske dokumenter. I Rstudio kan man lage
dynamiske dokumenter som blander tekst og R-kode. Et slikt dokument
består av ``text chunks'' og ``code chunks,'' altså bolker med både ren
tekst og koding.

\hypertarget{mulig-luxf8sning-r-notebooks}{%
\subsubsection{Mulig løsning R
Notebooks}\label{mulig-luxf8sning-r-notebooks}}

Daniel skrevet

\hypertarget{analyse}{%
\subsection{Analyse}\label{analyse}}

Med riktig bruk av R Notebook kan problemet med reproduserbarhet løses.
Dette dokumentet er skrevet i R studio, og det meste her er tekst, men
som tidligere nevnt er fordelen med R Notebook at man kan blande bolker
med tekst, sammen med bolker av koder. Når man laster ned pakker i
Rstudio får man med noen dataset, som man kan bruke til å øve seg. Et av
disse datasettene heter ``cars'' og code chunksene under henter data fra
dette settet. Den først koden viser hvor langt den lengste bilen i
datasettet kjørte.

\begin{Shaded}
\begin{Highlighting}[]
\FunctionTok{max}\NormalTok{(cars}\SpecialCharTok{$}\NormalTok{dist)}
\end{Highlighting}
\end{Shaded}

\begin{verbatim}
## [1] 120
\end{verbatim}

I følge Florian Markowetz (2015) er det følgende fem egoistiske
hovedgrunner til at forskerne selv burde ønske å publisere reproduserbar
forskning:

\begin{enumerate}
\def\labelenumi{\arabic{enumi}.}
\item
  Man unngår katastrofer

  \begin{itemize}
  \tightlist
  \item
    som replikasjonskrisen innenfor psykologi.
  \end{itemize}
\item
  Det er lettere å skrive artikler

  \begin{itemize}
  \tightlist
  \item
  \end{itemize}
\item
  Lettere for fagfeller å forstå tankegangen

  \begin{itemize}
  \tightlist
  \item
    blabla
  \end{itemize}
\item
  Det muliggjør kontinuitet i arbeidet

  \begin{itemize}
  \tightlist
  \item
    Det vil for eksempel ikke være noe stort problem dersom forskeren
    har glemt fremgangsmåten vedkommende brukte i forskningen sin i
    fjor. Det vil være muligheter for å kunne se hvordan man har tenkt
    og jobbet med studiet.
  \end{itemize}
\item
  Hjelper deg å opparbeide et godt rykte

  \begin{itemize}
  \tightlist
  \item
    Andre vil se på en forsker som publiserer reproduserbarforskning som
    en troverdig og grundig forsker, og dersom det noen gang blir
    problemer med noe av arbeidet, vil det være enkelt å vise og
    forklare hvordan man har tenkt og jobbet.
  \end{itemize}
\end{enumerate}

Er økt reproduserbarhet noe som vil tvinge seg frem eller er dagens økte
interesse bare et blaff? Kan reproduserbarhet ha relevans i sektorer
utenfor akademia?

\hypertarget{konklusjon}{%
\subsection{Konklusjon}\label{konklusjon}}

Vi kan konkludere med at RNotebook bidrar til å gjøre det mulig å
reprodusere, replikere og generalisere et studie, dette ved hjelp av en
dynamisk RMD fil som inneholder både data, koder, fremgangsmåte,
resultat og referanser, som igjen produserer docx-, tex- og
html-versjoner. Noe som kan by på hodebry og problemer er alle
programmene, extensions og Git som skal kommunisere sammen. Dette bidrar
til et uoversiktlig bilde i starten, og for de som skal ta det i bruk
krever det en bratt læringskurve. Det man kan trekke frem som positivt
for en forsker som skal ta dette i bruk er at man kan referere til
logikk og utregninger direkte i dokumentet. Det viser hva som ligger bak
og er ikke bare en visuell presentasjon. Om forskere da inkluderer alt
av data, koder, fremgangsmåte og full utredelse for hva som har blitt
gjort så vil dette kunne brukes av alle til å forstå og kunne brukes i
en senere studie, eller bare brukes som en referanse i en ny
studie/forskningsrapporter. Man bør ha flere retningslinjer og krav til
hva man bør inkludere når man publiserer nye
studier/rapporter/undersøkelser, dette vil bidra til økt standard for
fremtidig bruk.

\hypertarget{litteraturliste}{%
\subsection{Litteraturliste}\label{litteraturliste}}

Sætrevik, B. (2017). Replikasjonskrisen. \emph{Psykologtidsskriftet.}
\url{https://psykologtidsskriftet.no/fagessay/2017/07/replikasjonskrisen}

(\textbf{Bollen?})

(\textbf{Leek?})

(\textbf{Goodman?})

(\textbf{Schmidt?})

(\textbf{Gentleman?})

Bollen K., Cacioppo, J. T., Dean H., Kaplan R. M., Krosnick J. A. \&
Olds J. L. (2015). \emph{Social, Behavaioral, and Economic Sciences
Perspectives on Robust and Reliable Science.} (Report of the
Subcommittee on Replicability in Science Advisory Committee to the
National Science Foundation Directorate for Social, Behavioral, and
Economic Sciences).
\url{https://www.nsf.gov/sbe/AC_Materials/SBE_Robust_and_Reliable_Research_Report.pdf}

\hfill\break
Leek, J. T., \& Peng R. D. (2015) Opinion: Reproducible research can
still be wrong: Adopting a prevention approach. \emph{Proceedings of the
National Academy of Sciences, 112}(6), 1645-1646

Goodman, S. N., Fanelli, D. \& Ioannidis, J. P. A. (2016). What does
research reproducibility mean? \emph{Science Translational Medicine,}
8(341) 1-6.

Schmidt, M. L. 2015. Reproducible Research Using RMarkdown and Git
through Rstudio. \emph{RPubs by Rstudio.}
\url{https://rpubs.com/marschmi/105639}

Markowitz, F., Five selfish reasons to work reproducibly. \emph{Genome
Biology, 16}(1) 274.
\url{https://genomebiology.biomedcentral.com/articles/10.1186/s13059-015-0850-7}

Gentleman, R. \& Lang, D. T. Statistical analyses and reproducible
research. \emph{Journal of Computational and Graphical Statistics,
16}(1) 1-23.
\url{https://www.tandfonline.com/doi/abs/10.1198/106186007X178663}

(\textbf{Gentleman?})

\end{document}
